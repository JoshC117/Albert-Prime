\usepackage{tikz}
\usepackage[default]{lato}
\usepackage[T1]{fontenc}
\usepackage{titlesec}
\usepackage{titling}
\usepackage[hmargin=0.5in,bmargin=1in,tmargin=1in,centering]{geometry}
\usetikzlibrary{shadows.blur}
\usetikzlibrary{shapes.symbols}
\usetikzlibrary{positioning,fit,hobby}

% The red used throughout the document
\definecolor{red}{HTML}{D32D2D}
\pagenumbering{gobble}  

% Defines the red lines at the top and bottom of each page
\newcommand\AddLines{%
\begin{tikzpicture}[remember picture,overlay]
    \fill[red] (current page.north west) rectangle ([yshift=-2mm]current page.north east);
    \fill[red] (current page.south west) rectangle ([yshift=2mm]current page.south east);
    \end{tikzpicture}%
}

% Adds the red lines to each page
\AtBeginShipout{\AddLines}
\AtBeginShipoutFirst{\AddLines}

% Defines the red block for section titles
\newcommand\SecTitle[1]{%
\begin{tikzpicture}
  \node[inner ysep=1cm,text width=\paperwidth,fill=red,text=white,font=\Huge]  at (0,0) 
 {\parbox{0.4in}{\mbox{}}\parbox{\dimexpr\textwidth-0.4in\relax}{\raggedright\strut#1\strut}\parbox{0pt}{\mbox{}}};
\end{tikzpicture}%
}

% Adds the red block to section titles
\titleformat{\section}{\normalfont}{}{-0.5in}{\SecTitle}

% NO MORE INDENTS
\setlength\parindent{0pt}