\section{Specific Requirements}
	This section starts off by outlining the specific requirements for each external interface, followed by a description of the function requirements. It then continues with the performance requirements and design constraints, as well as the the attributes or non-functional requirements of the system. Finally, any other requirements not under the preceding categories are listed.
	\subsection{External Interface Requirements}
%	Detailed description for each system interfaces, user interfaces, hardware interfaces, software interfaces and communication interfaces. Include input and output, name, format, valid range, timing and other information.
	
	\subsection{Functional Requirements}
%	Detailed description of the functionality of each functional requirement. "The system shall do/perform/provide ...".	May include input validity checks, sequence of operations, responses to abnormal situations, input-output relationships.
	\textit{Empty for now.}
	
	\subsection{Performance Requirements}
%	Performance related capabilities of the product.
	\textit{Will be expanded upon.}\\
	\paragraph{User Interface}\\
	\begin{itemize}
	    \item The drag-and-drop system needs to render as quickly as possible, preferably in under 3 seconds.\\
	    \item The drag-and-drop system needs to avoid producing input lag by being too heavy on system resources.\\
	\end{itemize}
		
	\subsection{Design Constraints}
%	Describe all restrictions on the design alternatives such as standards or hardware limitations.	
	\textit{Empty for now.}
	
	\subsection{Software System Attributes}
%	Describe all quality-related requirements (reliability, security, availability, interoperability)
	\textit{Empty for now.}
	\paragraph{Reliability}
	\textit{Empty for now.}
	\paragraph{Security}
	\textit{Empty for now.}
	\paragraph{Availability}
	\textit{Empty for now.}
	\paragraph{Interoperability}
	\textit{Empty for now.}
	
	\subsection{Other Requirements}
	\textit{Empty for now.}
	
